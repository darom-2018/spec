\documentclass{scrartcl}

\usepackage{array}
\usepackage{bytefield}
\usepackage{fontspec}
\usepackage{gitinfo2}
\usepackage{hyperref}
\usepackage{microtype}
\usepackage{polyglossia}
\usepackage{scrlayer-scrpage}
\usepackage{xcolor}

\linespread{1.3}

\setmainlanguage{lithuanian}

\setmainfont{TeX Gyre Termes}
\setmonofont{TeX Gyre Cursor}


\ohead{Release \gitDescribe\ (\gitCommitterDate)}
\cfoot*{\pagemark}

\newcolumntype{?}{!{\vrule width 1pt}}

\definecolor{lightgray}{gray}{0.8}

\setlength{\headheight}{\baselineskip}
\setlength{\footheight}{\baselineskip}

\begin{document}
    \newcommand{\instr}[3]{\subparagraph{\makebox[6em][l]{\texttt{#1}}} (\texttt{#2})\par#3\par}
    \section{Reali mašina}
        \subsection{Procesorius}
            \subsubsection{Registrai}
                \paragraph{\makebox[3em][l]{PC}} Programų skaitiklis.
                \paragraph{\makebox[3em][l]{SP}} Steko nuoroda.
                \paragraph{\makebox[3em][l]{FLAGS}} Paskutinės aritmetinės ar loginės operacijos požymiai.
                \paragraph{\makebox[3em][l]{PTR}} Puslapių lentelės registras. \mbox {} \\
                    \par
                    \begin{bytefield}[endianness=big]{32}
                        \bitheader{0-31} \\
                        \bitbox{8}{$a_{0}$}
                        \bitbox{8}{$a_{1}$}
                        \bitbox{8}{$a_{2}$}
                        \bitbox{8}{$a_{3}$}
                    \end{bytefield}
                    \par
                    $a_{0}$ - programos dydis;
                    \par
                    $a_{1}$ - puslapių lentelės dydis;
                    \par
                    $a_{2}$ - bloko numeris, kuriame saugoma lentelė;
                    \par
                    $a_{3}$ - poslinkis bloke.
                \paragraph{\makebox[3em][l]{SHM}} Bendros atminties srities nuoroda.
                \paragraph{\makebox[3em][l]{PI}} Programinių pertraukimų registras.
                \paragraph{\makebox[3em][l]{SI}} Supervizorinių pertraukimų registras.
                \paragraph{\makebox[3em][l]{TI}} Taimerio registras.
        \subsection{Atmintis}
            Modelyje naudojami tokie dydžiai:
            \begin{itemize}
                \item žodis: du baitai;
                \item blokas: šešiolika žodžių.
            \end{itemize}
            \subsubsection{Supervizorinė atmintis}
                Supervizorinė atmintis skirta saugoti sisteminius procesus, kintamuosius, resursus ir mikroprogramas, interpretuojančias virtualios mašinos komandas. Modelyje supervizorinė atmintis nėra realizuojama; komandų vykdymą ir resursų valdymą atliks aukšto lygio kalbos procesorius. Supervizorinei atminčiai galima priskirti puslapių lenteles, kurioms išskirsime keturis blokus.
            \subsubsection{Vartotojo atmintis}
                Vartotojo atmintis skirta virtualių mašinų atmintims, jos dydis - 64 blokai (užtektinai keturioms maksimaliai didelėms virtualioms mašinoms).
            \subsubsection{Bendra atminties sritis}
                Virtualių mašinų reikmėms išskiriama bendra atminties sritis, kurioje mašinos gali dalintis duomenimis. Srities dydis - du blokai.
            \subsubsection{Išorinė atmintis}
                Išorinėje atmintyje saugomos programos. Realizuojama failu, kurio dydis dirbtinai neribojamas.
        \subsection{Kanalų įrenginys}
            Kanalų įrenginys yra atsakingas už duomenų persiuntimą tarp skirtingų mašinos komponentų. Įrenginio darbas organizuojamas nustatant specialius registrus:
            \begin{itemize}
                \item \textbf{SB} (Source Block) - bloko numeris, iš kurio bus kopijuojama;
                \item \textbf{DB} (Destination Block) - bloko numeris, į kurį bus kopijuojama;
                \item \textbf{SC} (Source Channel):
                    \subitem 1 - vartotojo atmintis;
                    \subitem 2 - supervizorinė atmintis;
                    \subitem 3 - išorinė atmintis;
                    \subitem 4 - standartinė įvestis;
                    \subitem 5 - LED lemputė;
                \item \textbf{DC} (Destination Channel):
                    \subitem 1 - vartotojo atmintis;
                    \subitem 2 - supervizorinė atmintis;
                    \subitem 3 - išorinė atmintis;
                    \subitem 4 - standartinė išvestis.
                    \subitem 5 - LED lemputė;                    
            \end{itemize}
            Kanalų įrenginys yra paslėptas nuo virtualios mašinos. Virtualios mašinos gali dirbti tik su virtualia atmintimi, virtualia klaviatūra ir monitoriumi. Darbui su virtualiais įvedimo ir išvedimo srautais yra naudojamos instrukcijos \texttt{IN} ir \texttt{OUT} (bei variantai). Šios instrukcijos atitinkamai nustato kanalų įrenginio registrų reikšmes ir blokuoja tolesnį vykdymą tuo atveju, kai kanalas yra užimtas.
            \subsubsection{LED lemputė}
            	LED lemputė - įrenginys, turintis vieną bloką atminties, kuriame saugomos tris reikšmės. Tos reikšmės atitinką spalvos, kurią dabar atkuria lemputė, RGB kodą. Jeigu savo programoje norėtume išjungti lemputę, turėtume nustatyti visas tris RGB reikšmes į 0. Darbui su lempute naudojama komanda
            	\texttt{LED}.
    \pagebreak
    \section{Virtuali mašina}
        \subsection{Procesorius}
            \subsubsection{Registrai}
                \paragraph{\makebox[4em][l]{PC}} Programų skaitiklis.
                \paragraph{\makebox[4em][l]{SP}} Steko nuoroda.
                \paragraph{\makebox[4em][l]{FLAGS}} Paskutinės aritmetinės ar loginės operacijos požymiai. \mbox{} \\
                    \par
                    \begin{bytefield}[bitwidth=1.5em,endianness=big]{16}
                        \bitheader{0-15} \\
                        \bitbox{7}{\color{lightgray}\rule{\width}{\height}} & \bitbox{1}{CF}
                        \bitbox{3}{\color{lightgray}\rule{\width}{\height}} & \bitbox{1}{PF}
                        \bitbox{3}{\color{lightgray}\rule{\width}{\height}} & \bitbox{1}{ZF}
                    \end{bytefield}
            \subsubsection{Instrukcijų rinkinys}
                Kur nurodyta, steko argumentai išvardinti tvarka, kuria yra skaitomi iš steko.
                \paragraph{Bendros paskirties operacijos}
                    \instr{NOP}{00 00}{Neatlieka jokio veiksmo.}
                    \instr{HALT}{00 01}{Sustabdo procesoriaus darbą (nutraukia programos vykdymą).}
                \paragraph{Steko operacijos}
                    \instr{DUP}{01 00}{Padaro steko viršūnės kopiją.}
                    \instr{POP}{01 01}{Nukopijuoja steko žodį į virtualios mašinos atmintį.}
                    \emph{Steko argumentai:} žodžio numeris virtualios mašinos atmintyje, bloko numeris virtualios mašinos atmintyje, kopijuojamas žodis.
                    \instr{PUSH imm16}{01 02}{Nukopijuoja operando žodį į steką.}
                    \instr{PUSHM}{01 03}{Į steką iš atminties nukopijuoja žodį.}
                    \emph{Steko argumentai:} žodžio numeris virtualios mašinos atmintyje, bloko numeris virtualios mašinos atmintyje.
                    \instr{PUSHF}{01 04}{Nukopijuoja FLAGS registro reikšmę į steką.}
                \paragraph{Aritmetinės operacijos}
                    \instr{ADD}{02 00}{Sudeda du steke esančius žodžius ir patalpina rezultatą steko viršūnėje.}
                    \instr{CMP}{02 01}{Palygina du steke esančius žodžius.}
                    \instr{DEC}{02 02}{Steko viršūnėje esančio žodžio reikšmę sumažina vienetu.}
                    \instr{DIV}{02 03}{Padalina pirmąjį steke esantį žodį iš antrojo ir patalpina rezultatą steko viršūnėje.}
                    \instr{INC}{02 04}{Steko viršūnėje esančio žodžio reikšmę padidina vienetu.}
                    \instr{MUL}{02 05}{Sudaugina du steke esančius žodžius ir patalpina rezultatą steko viršūnėje.}
                    \instr{SUB}{02 06}{Atima antrąjį steke esantį žodį iš pirmojo ir patalpina rezultatą steko viršūnėje.}
                \paragraph{Loginės operacijos}
                    \instr{AND}{03 00}{Atlieka dviejų steke esančių žodžių konjunkciją ir patalpina rezultatą steko viršūnėje.}
                    \instr{NOT}{03 01}{Atlieka steko viršūnėje esančio žodžio inversiją.}
                    \instr{OR}{03 02}{Atlieka dviejų steke esančių žodžių disjunkciją ir patalpina rezultatą steko viršūnėje.}
                    \instr{XOR}{03 03}{Atlieka dviejų steke esančių žodžių griežtą disjunkciją ir patalpina rezultatą steko viršūnėje.}
                \paragraph{Valdymo operacijos}
                    \instr{JMP}{04 00}{Besąlygiškai atlieka šuolį į adresą.}
                    \emph{Steko argumentai:} žodžio numeris virtualios mašinos atmintyje, bloko numeris virtualios mašinos atmintyje.
                    \instr{JC}{04 01}{Atlieka šuolį į adresą, kai \texttt{CF = 1}.}
                    \emph{Steko argumentai:} žodžio numeris virtualios mašinos atmintyje, bloko numeris virtualios mašinos atmintyje.
                    \instr{JE}{04 02}{Atlieka šuolį į adresą, kai \texttt{ZF = 1}.}
                    \emph{Steko argumentai:} žodžio numeris virtualios mašinos atmintyje, bloko numeris virtualios mašinos atmintyje.
                    \instr{JG}{04 03}{Atlieka šuolį į adresą, kai \texttt{ZF = 0} ir \texttt{CF = 1}.}
                    \emph{Steko argumentai:} žodžio numeris virtualios mašinos atmintyje, bloko numeris virtualios mašinos atmintyje.
                    \instr{JGE}{04 04}{Atlieka šuolį į adresą, kai \texttt{ZF = 0} ir \texttt{CF = 1} arba \texttt{ZF = 1}.}
                    \emph{Steko argumentai:} žodžio numeris virtualios mašinos atmintyje, bloko numeris virtualios mašinos atmintyje.
                    \instr{JL}{04 05}{Atlieka šuolį į adresą, kai \texttt{ZF = 0} ir \texttt{CF = 0}.}
                    \emph{Steko argumentai:} žodžio numeris virtualios mašinos atmintyje, bloko numeris virtualios mašinos atmintyje.
                    \instr{JLE}{04 06}{Atlieka šuolį į adresą, kai \texttt{ZF = 0} ir \texttt{CF = 0} arba \texttt{ZF = 1}.}
                    \emph{Steko argumentai:} žodžio numeris virtualios mašinos atmintyje, bloko numeris virtualios mašinos atmintyje.
                    \instr{JNC}{04 07}{Atlieka šuolį į adresą, kai \texttt{CF = 0}.}
                    \emph{Steko argumentai:} žodžio numeris virtualios mašinos atmintyje, bloko numeris virtualios mašinos atmintyje.
                    \instr{JNE}{04 08}{Atlieka šuolį į adresą, kai \texttt{ZF = 0}.}
                    \emph{Steko argumentai:} žodžio numeris virtualios mašinos atmintyje, bloko numeris virtualios mašinos atmintyje.
                    \instr{JNP}{04 09}{Atlieka šuolį į adresą, kai \texttt{PF = 0}.}
                    \emph{Steko argumentai:} žodžio numeris virtualios mašinos atmintyje, bloko numeris virtualios mašinos atmintyje.
                    \instr{JP}{04 0A}{Atlieka šuolį į adresą, kai \texttt{PF = 1}.}
                    \emph{Steko argumentai:} žodžio numeris virtualios mašinos atmintyje, bloko numeris virtualios mašinos atmintyje.
                    \instr{LOOP}{04 0B}{Atlieka šuolį į adresą, jei skaitiklio reikšmė didesnė už nulį.}
                    \emph{Steko argumentai:} žodžio numeris virtualios mašinos atmintyje, bloko numeris virtualios mašinos atmintyje, skaitiklis.
                \paragraph{Įvesties/išvesties operacijos}
                    \instr{IN}{05 00}{Iš įvesties srauto į virtualios mašinos atmintį nukopijuoja ASCII koduote užkoduotus duomenis.}
                    \emph{Steko argumentai:} žodžių skaičius, žodžio numeris virtualios mašinos atmintyje, bloko numeris virtualios mašinos atmintyje.
                    \instr{OUT}{05 01}{Į išvesties srautą nusiunčia ASCII koduote užkoduotus duomenis.}
                    \emph{Steko argumentai:} žodžių skaičius, žodžio numeris virtualios mašinos atmintyje, bloko numeris virtualios mašinos atmintyje.
                    \instr{OUTI}{05 02}{Į išvesties srautą skaitiniu pavidalu nusiunčia žodį iš steko viršūnės.}
                \paragraph{Darbo su bendra atminties sritimi operacijos}
                    \instr{SHREAD}{06 00}{Iš bendros atminties srities nukopijuoja duomenis į virtualios mašinos atmintį.}
                    \emph{Steko argumentai:} žodžių skaičius, žodžio numeris virtualios mašinos atmintyje, bloko numeris virtualios mašinos atmintyje, žodžio numeris bendroje atminties srityje, bloko numeris bendroje atminties srityje.
                    \instr{SHWRITE}{06 01}{Iš virtualios mašinos nukopijuoja duomenis į bendrą atminties sritį.}
                    \emph{Steko argumentai:} žodžių skaičius, žodžio numeris virtualios mašinos atmintyje, bloko numeris virtualios mašinos atmintyje, žodžio numeris bendroje atminties srityje, bloko numeris bendroje atminties srityje.
                \paragraph{Papildomo įrenginio valdymas}
					\instr{LED}{07 00}{Nustato lemputės spalvą}
					\emph{Steko argumentai:}{Red Green ir Blue spalvų vertės. (0,0,0) jei norime išjungti lemputę}
        \subsection{Atmintis}
            Virtualios mašinos atmintį sudaro šešiolika vartotojo atminties blokų, kurie yra suskirstyti į segmentus pačios programos. Naudojami segmentai:
            \begin{itemize}
                \item kodo,
                \item duomenų,
                \item steko.
            \end{itemize}
            \subsubsection{Puslapiavimo mechanizmas}
                Puslapiavimo mechanizmas įgyvendinamas remiantis dviem dalykais: realios mašinos procesoriaus registru \texttt{PTR} bei virtualių mašinų puslapių lentelėmis.
		\subsection{Užduoties pavyzdys}
			\begin{itemize}
				\item[] \$PROGRAM	
				\item[] COUNTDOWN	\emph{//Užduoties pavadinimas}
				\item [0] PUSH 0010	\emph{//Į steką įdedame reikšmė 10}
				\item [4] OUTI		\emph{//Viršutinė steko reikšmė išvedama į ekraną}
				\item [6] DEC		\emph{//Viršutinė steko reikšmė sumažinama vienetu}
				\item [8] PUSH 0000	
				\item [12] PUSH 0004
				\item [16] JNE		\emph{//Jei ZF != 0 valdymas perduodamas komandai, kurios žodžio ir bloko numeriai yra steke}
				\item [18] HALT		
				\item[] \$END
			\end{itemize}
		
\end{document}
