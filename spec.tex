\documentclass{scrartcl}

\usepackage{array}
\usepackage{bytefield}
\usepackage{fontspec}
\usepackage{gitinfo2}
\usepackage{hyperref}
\usepackage{microtype}
\usepackage{scrlayer-scrpage}
\usepackage{xcolor}

\setmainfont{TeX Gyre Termes}
\setmonofont{TeX Gyre Cursor}

\ohead{Release \gitDescribe\ (\gitCommitterDate)}
\cfoot*{\pagemark}

\newcolumntype{?}{!{\vrule width 1pt}}

\definecolor{lightgray}{gray}{0.8}

\begin{document}
\newcommand{\instr}[3]{\subparagraph{\makebox[6em][l]{\texttt{#1}}} (\texttt{#2})\par#3\par}

    \section{Reali mašina}
    \section{Virtuali mašina}
        \subsection{Procesorius}
            \subsubsection{Registrai}
                \paragraph{PC} \mbox{} \\
                    Dviejų baitų dydžio registras, kuriame saugomas adresas instrukcijos, kuri bus vykdoma po dabartinės.
                \paragraph{SP} \mbox{} \\
                    Dviejų baitų dydžio registras, kuriame saugomas steko viršūnės adresas.
                \paragraph{FLAGS} \mbox{} \\
                    Dviejų baitų dydžio registras, kuriame saugomi paskutinės aritmetinės ar loginės operacijos požymiai.
                    \begin{center}
                        \begin{bytefield}[bitwidth=1.5em,endianness=big]{16}
                            \bitheader{0-15} \\
                            \bitbox{3}{\color{lightgray}\rule{\width}{\height}} & \bitbox{1}{CF}
                            \bitbox{3}{\color{lightgray}\rule{\width}{\height}} & \bitbox{1}{OF}
                            \bitbox{3}{\color{lightgray}\rule{\width}{\height}} & \bitbox{1}{PF}
                            \bitbox{3}{\color{lightgray}\rule{\width}{\height}} & \bitbox{1}{ZF}
                        \end{bytefield}
                    \end{center}
            \subsubsection{Instrukcijų rinkinys}
                \paragraph{Bendros paskirties operacijos}
                    \instr{NOP}{00 00}{Neatlieka jokio veiksmo.}
                    \instr{HALT}{00 01}{Sustabdo procesoriaus darbą (nutraukia programos vykdymą).}
                \paragraph{Steko operacijos}
                    \instr{DUP}{01 00}{Padaro steko viršūnės kopiją.}
                    Pseudokodas:
\begin{verbatim}
     SP  :=  SP + 2
    [SP] := [SP - 2]
\end{verbatim}
                    \instr{POP m16}{01 01}{Nukopijuoja steko viršūnę į atminties vietą, nurodytą operande.}
                    Pseudokodas:
\begin{verbatim}
    [OP] := [  SP  ]
     SP  :=  SP - 2
\end{verbatim}
                    \instr{PUSH m16}{01 02}{Nukopijuoja žodį iš operande nurodytos vietos atmintyje į steką.}
                    Pseudokodas:
\begin{verbatim}
     SP  :=  SP + 2
    [SP] := [  OP  ]
\end{verbatim}
                    \instr{PUSH imm16}{01 03}{Nukopijuoja operando žodį į steką.}
                    Pseudokodas:
\begin{verbatim}
     SP  := SP + 2
    [SP] := OP
\end{verbatim}
                    \instr{PUSHF}{01 04}{Nukopijuoja FLAGS registro reikšmę į steką.}
                    Pseudokodas:
\begin{verbatim}
     SP  :=  SP + 2
    [SP] := [FLAGS ]
\end{verbatim}
                \paragraph{Aritmetinės operacijos}
                    \instr{ADD}{02 00}{Sudeda du steke esančius žodžius ir patalpina rezultatą steko viršūnėje.}
                    \instr{CMP}{02 01}{Palygina du steke esančius žodžius.}
                    \instr{DEC}{02 02}{Steko viršūnėje esančio žodžio reikšmę sumažina vienetu.}
                    \instr{DIV}{02 03}{Padalina pirmąjį steke esantį žodį iš antrojo ir patalpina rezultatą steko viršūnėje.}
                    \instr{INC}{02 04}{Steko viršūnėje esančio žodžio reikšmę padidina vienetu.}
                    \instr{MUL}{02 05}{Sudaugina du steke esančius žodžius ir patalpina rezultatą steko viršūnėje.}
                    \instr{SUB}{02 06}{Atima antrąjį steke esantį žodį iš pirmojo ir patalpina rezultatą steko viršūnėje.}
                \paragraph{Loginės operacijos}
                    \instr{AND}{03 00}{Atlieka dviejų steke esančių žodžių konjunkciją ir patalpina rezultatą steko viršūnėje.}
                    \instr{NOT}{03 01}{Atlieka steko viršūnėje esančio žodžio inversiją.}
                    \instr{OR}{03 02}{Atlieka dviejų steke esančių žodžių disjunkciją ir patalpina rezultatą steko viršūnėje.}
                    \instr{XOR}{03 03}{Atlieka dviejų steke esančių žodžių griežtą disjunkciją ir patalpina rezultatą steko viršūnėje.}
                \paragraph{Valdymo operacijos}
                    \instr{JMP}{04 00}{Besąlygiškai perduoda valdymą steko viršūnėje nurodytu adresu.}
                    \instr{JC}{04 01}{Perduoda valdymą adresu, nurodytu steko viršūnėje, kai \texttt{CF = 1}.}
                    \instr{JE}{04 02}{Perduoda valdymą adresu, nurodytu steko viršūnėje, kai \texttt{ZF = 1}.}
                    \instr{JG}{04 03}{Perduoda valdymą adresu, nurodytu steko viršūnėje, kai \texttt{ZF = 0} ir \texttt{OF = 1}.}
                    \instr{JGE}{04 04}{Perduoda valdymą adresu, nurodytu steko viršūnėje, kai \texttt{ZF = 0} ir \texttt{OF = 1} arba \texttt{ZF = 1}.}
                    \instr{JL}{04 05}{Perduoda valdymą adresu, nurodytu steko viršūnėje, kai \texttt{ZF = 0} ir \texttt{OF = 0}.}
                    \instr{JLE}{04 06}{Perduoda valdymą adresu, nurodytu steko viršūnėje, kai \texttt{ZF = 0} ir \texttt{OF = 0} arba \texttt{ZF = 1}.}
                    \instr{JNC}{04 07}{Perduoda valdymą adresu, nurodytu steko viršūnėje, kai \texttt{CF = 0}.}
                    \instr{JNE}{04 08}{Perduoda valdymą adresu, nurodytu steko viršūnėje, kai \texttt{ZF = 0}.}
                    \instr{JNP}{04 09}{Perduoda valdymą adresu, nurodytu steko viršūnėje, kai \texttt{PF = 0}.}
                    \instr{JP}{04 0A}{Perduoda valdymą adresu, nurodytu steko viršūnėje, kai \texttt{PF = 1}.}
                    \instr{LOOP}{04 0B}{}
                \paragraph{Įvesties/išvesties operacijos}
                    \instr{IN}{05 00}{Iš įvesties srauto steko viršūnėje nurodytu adresu įrašo šešiolikos žodžių ilgio bloką.}
                    \instr{OUT}{05 01}{Į išvesties srautą nusiunčia steko viršūnėje nurodytame adrese esantį šešiolikos žodžių ilgio atminties bloką.}
                \paragraph{Darbo su atmintimi operacijos}
                    \instr{MOV}{06 00}{Pirmąjame steko viršūnės žodyje nurodytu adresu įrašo antrąjį steko viršūnės žodį.}
                    \instr{SHREAD}{06 01}{Steko viršūnėje nurodytu adresu nukopijuoja bendros atminties srities turinį.}
                    \instr{SHWRITE}{06 02}{Į bendrą atminties sritį įrašo steko viršūnėje nurodytame adrese esantį bloką.}
\end{document}
