\documentclass{scrartcl}

\usepackage{array}
\usepackage{bytefield}
\usepackage{fontspec}
\usepackage{hyperref}
\usepackage{microtype}
\usepackage{scrlayer-scrpage}
\usepackage{xcolor}

\setmainfont{TeX Gyre Termes}
\setmonofont{TeX Gyre Cursor}

\ohead{}
\cfoot*{\pagemark}

\newcolumntype{?}{!{\vrule width 1pt}}

\definecolor{lightgray}{gray}{0.8}

\begin{document}
    \section{Reali mašina}

    \pagebreak

    \section{Virtuali mašina}
        \subsection{Procesorius}
            \subsubsection{Registrai}
                \paragraph{PC}
                    Dviejų baitų dydžio registras, kuriame saugomas adresas instrukcijos, kuri bus vykdoma po dabartinės.
                \paragraph{SP}
                    Dviejų baitų dydžio registras, kuriame saugomas steko viršūnės adresas.
                \paragraph{FLAGS}
                    Dviejų baitų dydžio registras, kuriame saugomi paskutinės aritmetinės ar loginės operacijos požymiai.

                    \vspace{1em}

                    \begin{bytefield}[bitwidth=1.5em,endianness=big]{16}
                        \bitheader{0-15}  \\
                        \bitbox{3}{\color{lightgray}\rule{\width}{\height}} & \bitbox{1}{CF}
                        \bitbox{3}{\color{lightgray}\rule{\width}{\height}} & \bitbox{1}{OF}
                        \bitbox{3}{\color{lightgray}\rule{\width}{\height}} & \bitbox{1}{PF}
                        \bitbox{3}{\color{lightgray}\rule{\width}{\height}} & \bitbox{1}{ZF}
                    \end{bytefield}
            \subsubsection{Instrukcijų rinkinys}
                \paragraph{Bendros paskirties operacijos}
                    \vspace{1em}

                    \begin{center}
                        \begin{tabular}{? p{3cm} ? l ? p{8cm} ?}
                            \noalign{\hrule height 1pt}
                            Mnemonika & Kodas & Aprašymas \\
                            \noalign{\hrule height 1pt}
                            \texttt{NOP}  & \texttt{00 00} & Neatlieka jokio veiksmo                                   \\
                            \hline
                            \texttt{HALT} & \texttt{00 01} & Sustabdo procesoriaus darbą (nutraukia programos vykdymą) \\
                            \noalign{\hrule height 1pt}
                        \end{tabular}
                    \end{center}

                \paragraph{Steko operacijos}
                    \vspace{1em}

                    \begin{center}
                        \begin{tabular}{? p{3cm} ? l ? p{8cm} ?}
                            \noalign{\hrule height 1pt}
                            Mnemonika  & Kodas          & Aprašymas                                                       \\
                            \noalign{\hrule height 1pt}
                            \texttt{DUP}        & \texttt{01 00} & Padaro steko viršūnės kopiją                                    \\
                            \hline
                            \texttt{POP m16}    & \texttt{01 01} & Nukopijuoja steko viršūnę į atminties vietą, nurodytą operande  \\
                            \hline
                            \texttt{PUSH m16}   & \texttt{01 02} & Nukopijuoja žodį iš operande nurodytos vietos atmintyje į steką \\
                            \hline
                            \texttt{PUSH imm16} & \texttt{01 03} & Nukopijuoja operando žodį į steką                               \\
                            \hline
                            \texttt{PUSHF}      & \texttt{01 04} & Nukopijuoja FLAGS registro reikšmę į steką                      \\
                            \noalign{\hrule height 1pt}
                        \end{tabular}
                    \end{center}

                \paragraph{Aritmetinės operacijos}
                    \vspace{1em}

                    \begin{center}
                        \begin{tabular}{? p{3cm} ? l ? p{8cm} ?}
                            \noalign{\hrule height 1pt}
                            Mnemonika  & Kodas          & Aprašymas                                                       \\
                            \noalign{\hrule height 1pt}
                            \texttt{ADD} & \texttt{02 00} & Sudeda du steke esančius skaičius ir rezultatą patalpina steko viršūnėje \\
                            \hline
                            \texttt{DIV} & \texttt{02 01} & Padalina pirmąjį steke esantį skaičių iš antrojo ir rezultatą patalpina steko viršūnėje \\
                            \hline
                            \texttt{MUL} & \texttt{02 02} & Nukopijuoja žodį iš operande nurodytos vietos atmintyje į steką \\
                            \hline
                            \texttt{SUB} & \texttt{02 03} & Nukopijuoja operando žodį į steką                               \\
                            \noalign{\hrule height 1pt}
                        \end{tabular}
                    \end{center}
\end{document}
