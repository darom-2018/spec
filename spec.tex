\documentclass{scrartcl}

\usepackage{array}
\usepackage{bytefield}
\usepackage{fontspec}
\usepackage{gitinfo2}
\usepackage{hyperref}
\usepackage{microtype}
\usepackage{scrlayer-scrpage}
\usepackage{xcolor}

\setmainfont{TeX Gyre Termes}
\setmonofont{TeX Gyre Cursor}

\ohead{Release \gitDescribe\ (\gitCommitterDate)}
\cfoot*{\pagemark}

\newcolumntype{?}{!{\vrule width 1pt}}

\definecolor{lightgray}{gray}{0.8}

\begin{document}
    \section{Reali mašina}
    \pagebreak
    \section{Virtuali mašina}
        \subsection{Procesorius}
            \subsubsection{Registrai}
                \paragraph{PC}
                    Dviejų baitų dydžio registras, kuriame saugomas adresas instrukcijos, kuri bus vykdoma po dabartinės.
                \paragraph{SP}
                    Dviejų baitų dydžio registras, kuriame saugomas steko viršūnės adresas.
                \paragraph{FLAGS}
                    Dviejų baitų dydžio registras, kuriame saugomi paskutinės aritmetinės ar loginės operacijos požymiai.

                    \vspace{1em}

                    \begin{bytefield}[bitwidth=1.5em,endianness=big]{16}
                        \bitheader{0-15}  \\
                        \bitbox{3}{\color{lightgray}\rule{\width}{\height}} & \bitbox{1}{CF}
                        \bitbox{3}{\color{lightgray}\rule{\width}{\height}} & \bitbox{1}{OF}
                        \bitbox{3}{\color{lightgray}\rule{\width}{\height}} & \bitbox{1}{PF}
                        \bitbox{3}{\color{lightgray}\rule{\width}{\height}} & \bitbox{1}{ZF}
                    \end{bytefield}
            \subsubsection{Instrukcijų rinkinys}
                \paragraph{Bendros paskirties operacijos}
                    \begin{center}
                        \begin{tabular}{? p{3cm} ? l ? p{8cm} ?}
                            \noalign{\hrule height 1pt}
                            Mnemonika     & Kodas & Aprašymas                                                          \\
                            \noalign{\hrule height 1pt}
                            \texttt{NOP}  & \texttt{00 00} & Neatlieka jokio veiksmo                                   \\
                            \hline
                            \texttt{HALT} & \texttt{00 01} & Sustabdo procesoriaus darbą (nutraukia programos vykdymą) \\
                            \noalign{\hrule height 1pt}
                        \end{tabular}
                    \end{center}
                \paragraph{Steko operacijos}
                    \begin{center}
                        \begin{tabular}{? p{3cm} ? l ? p{8cm} ?}
                            \noalign{\hrule height 1pt}
                            Mnemonika           & Kodas          & Aprašymas                                                       \\
                            \noalign{\hrule height 1pt}
                            \texttt{DUP}        & \texttt{01 00} & Padaro steko viršūnės kopiją                                    \\
                            \hline
                            \texttt{POP m16}    & \texttt{01 01} & Nukopijuoja steko viršūnę į atminties vietą, nurodytą operande  \\
                            \hline
                            \texttt{PUSH m16}   & \texttt{01 02} & Nukopijuoja žodį iš operande nurodytos vietos atmintyje į steką \\
                            \hline
                            \texttt{PUSH imm16} & \texttt{01 03} & Nukopijuoja operando žodį į steką                               \\
                            \hline
                            \texttt{PUSHF}      & \texttt{01 04} & Nukopijuoja FLAGS registro reikšmę į steką                      \\
                            \noalign{\hrule height 1pt}
                        \end{tabular}
                    \end{center}
                \pagebreak
                \paragraph{Aritmetinės operacijos}
                    \begin{center}
                        \begin{tabular}{? p{3cm} ? l ? p{8cm} ?}
                            \noalign{\hrule height 1pt}
                            Mnemonika    & Kodas          & Aprašymas                                                                               \\
                            \noalign{\hrule height 1pt}
                            \texttt{ADD} & \texttt{02 00} & Sudeda du steke esančius žodžius ir patalpina rezultatą steko viršūnėje                 \\
                            \hline
                            \texttt{DEC} & \texttt{02 01} & Steko viršūnėje esančio žodžio reikšmę sumažina vienetu                                 \\
                            \hline
                            \texttt{DIV} & \texttt{02 02} & Padalina pirmąjį steke esantį žodį iš antrojo ir patalpina rezultatą steko viršūnėje    \\
                            \hline
                            \texttt{INC} & \texttt{02 03} & Steko viršūnėje esančio žodžio reikšmę padidina vienetu                                 \\
                            \hline
                            \texttt{MUL} & \texttt{02 04} & Sudaugina du steke esančius žodžius ir patalpina rezultatą steko viršūnėje              \\
                            \hline
                            \texttt{SUB} & \texttt{02 05} & Atima antrąjį steke esantį žodį iš pirmojo ir patalpina rezultatą steko viršūnėje       \\
                            \noalign{\hrule height 1pt}
                        \end{tabular}
                    \end{center}
                \paragraph{Loginės operacijos}
                    \begin{center}
                        \begin{tabular}{? p{3cm} ? l ? p{8cm} ?}
                            \noalign{\hrule height 1pt}
                            Mnemonika    & Kodas          & Aprašymas                                                                                      \\
                            \noalign{\hrule height 1pt}
                            \texttt{AND} & \texttt{03 00} & Atlieka dviejų steke esančių žodžių konjunkciją ir patalpina rezultatą steko viršūnėje         \\
                            \hline
                            \texttt{NOT} & \texttt{03 01} & Atlieka steko viršūnėje esančio žodžio inversiją                                               \\
                            \hline
                            \texttt{OR}  & \texttt{03 02} & Atlieka dviejų steke esančių žodžių disjunkciją ir patalpina rezultatą steko viršūnėje         \\
                            \hline
                            \texttt{XOR} & \texttt{03 03} & Atlieka dviejų steke esančių žodžių griežtą disjunkciją ir patalpina rezultatą steko viršūnėje \\
                            \noalign{\hrule height 1pt}
                        \end{tabular}
                    \end{center}
                \paragraph{Valdymo operacijos}
                    \begin{center}
                        \begin{tabular}{? p{3cm} ? l ? p{8cm} ?}
                            \noalign{\hrule height 1pt}
                            Mnemonika    & Kodas          & Aprašymas \\
                            \noalign{\hrule height 1pt}
                            \texttt{JMP}  & \texttt{04 00} & \\
                            \hline
                            \texttt{JC}   & \texttt{04 01} & \\
                            \hline
                            \texttt{JE}   & \texttt{04 02} & \\
                            \hline
                            \texttt{JG}   & \texttt{04 03} & \\
                            \hline
                            \texttt{JGE}  & \texttt{04 04} & \\
                            \hline
                            \texttt{JL}   & \texttt{04 05} & \\
                            \hline
                            \texttt{JLE}  & \texttt{04 06} & \\
                            \hline
                            \texttt{JNC}  & \texttt{04 07} & \\
                            \hline
                            \texttt{JNE}  & \texttt{04 08} & \\
                            \hline
                            \texttt{JNP}  & \texttt{04 09} & \\
                            \hline
                            \texttt{JP}   & \texttt{04 0A} & \\
                            \hline
                            \texttt{LOOP} & \texttt{04 0B} & \\
                            \noalign{\hrule height 1pt}
                        \end{tabular}
                    \end{center}
                \paragraph{Įvesties/išvesties operacijos}
                    \begin{center}
                        \begin{tabular}{? p{3cm} ? l ? p{8cm} ?}
                            \noalign{\hrule height 1pt}
                            Mnemonika    & Kodas          & Aprašymas                                                                                                     \\
                            \noalign{\hrule height 1pt}
                            \texttt{IN}  & \texttt{05 00} & Iš įvesties srauto steko viršūnėje nurodytu adresu įrašo šešiolikos žodžių ilgio bloką                        \\
                            \hline
                            \texttt{OUT} & \texttt{05 01} & Į išvesties srautą nusiunčia steko viršūnėje nurodytame adrese esantį šešiolikos žodžių ilgio atminties bloką \\
                            \noalign{\hrule height 1pt}
                        \end{tabular}
                    \end{center}
                \paragraph{Darbo su atmintimi operacijos}
                    \begin{center}
                        \begin{tabular}{? p{3cm} ? l ? p{8cm} ?}
                            \noalign{\hrule height 1pt}
                            Mnemonika        & Kodas          & Aprašymas \\
                            \noalign{\hrule height 1pt}
                            \texttt{MOV}     & \texttt{06 00} & Pirmąjame steko viršūnės žodyje nurodytu adresu įrašo antrąjį steko viršūnės žodį \\
                            \hline
                            \texttt{SHREAD}  & \texttt{06 01} & Steko viršūnėje nurodytu adresu nukopijuoja bendros atminties srities turinį      \\
                            \hline
                            \texttt{SHWRITE} & \texttt{06 02} & Į bendrą atminties sritį įrašo steko viršūnėje nurodytame adrese esantį bloką     \\
                            \noalign{\hrule height 1pt}
                        \end{tabular}
                    \end{center}
\end{document}
